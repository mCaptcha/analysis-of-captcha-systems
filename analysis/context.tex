\subsection{Context-based}

\subsubsection{Privacy}
Excellent\\
The method doesn't employ any tracking elements and works when used in anonymous
networks like TOR\@.

\subsubsection{Effectiveness}
Bad\\

Domain context is limited in nature. An attacker could either become familiar
with the service and then mount an attack, or they could mount brute force attack
to aggregate all possible challenges that the service presents.

Both of these attacks are practical.

% TODO cite CAPTCHA farm cost analysis paper
ease.
\subsubsection{Accessibility}
Bad\\

This method poses challenges to users with cognitive disabilities as it requires
them to retain information and recollect in a timely manner. Also, this method
makes poses challenges to new visitors to service. The familiarization period to
solve CAPTCHAs in a timely fashion might be too high for someone new.

\subsubsection{Accuracy}
Bad\\
Success and failure are absolute states in this method. A solution that doesn't
match the challenge text results in failure while a match is considered a
success. But when above-mentioned circumvention methods are used, there will be
results will be completely inaccurate and as the system lacks any adaptational
capabilities, the failure will be long-lasting.  

The method uses only challenge answers in its decision process. No other
external factors are involved.
