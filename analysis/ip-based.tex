\subsection{IP tracking-based}

\subsubsection{Privacy}
Bad\\
IP tracking poses privacy threats to users who prefer to be anonymous. Also,
when accessing a service via a VPN like TOR, IP tracking produces false
positives.

\subsubsection{Effectiveness}
Bad\\

Infected computers around the world are used as botnets. So attackers have
access to a wide range of burnable IP addresses. In such cases, IP tracking-based   
solutions result in total failure.

Also, due to IPv4 exhaustion and the slow adoption of IPv6, several users access
the internet through Network Address Translation (NAT) routing. In such cases, a
single IP can represent thousands of users. If the system relies on IP
based-tracking entirely for its decision making process, then it will have to
introduce relaxation rules to accommodate users behind NAT, which reduces it's
effectiveness.

Both of these attacks are practical.

% TODO cite CAPTCHA farm cost analysis paper
ease.
\subsubsection{Accessibility}
Bad\\

Without relaxation rules for NAT users, unassuming users will be falsely flagged as
malicious and will dramatically affect their accessibility to the service.

\subsubsection{Accuracy}
Bad\\

This method does not produce accurate results when it encounters users behind
NATs or VPNs.
