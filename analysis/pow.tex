\subsection{Proof of Work}

\subsubsection{Privacy}
Excellent\\
Proof of Work (PoW) doesn't use any tracking elements and are able to work
in anonymous networks like TOR\@.

\subsubsection{Effectiveness}
Excellent\\

Proofs are cryptographically sound and can't be forged. PoW works on the idea
that the work done to send a request must be more than the work done to respond
to it. Therefore, a successful attack will require the attacker to dedicate
significantly more resources than what the service provider uses to run the
service.

% TODO cite CAPTCHA farm cost analysis paper
ease.
\subsubsection{Accessibility}
Good\\

The process is fully automated so doesn't require any user interactions. So it
is ideal for users with auditory, cognitive and visual disabilities. But it
poses challenges to users with slower devices. Some PoW implementations
time-to-live (TTL) on challenges so when a device is not able to generate proofs
within that period, their solution will be rejected and will be bared from
accessing the service.

\subsubsection{Accuracy}
Good\\

Success and failure are absolute states in this method. A proof that doesn't
pass verification will be rejected and the user barred from accessing the
service. But when above-mentioned circumvention methods are used, there will be
results will be completely inaccurate and as the system lacks any adaptational
capabilities, the failure will be long-lasting.  

The method uses only challenge proofs in its decision process. No other
external factors are involved.
