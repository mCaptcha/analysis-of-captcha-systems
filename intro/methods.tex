\subsection{CAPTCHA methods analysed}
We analysed at the following CAPTCHA methods using the above-mentioned
parameters. These are popular methods are currently in deployment.
%TODO add images

\subsubsection{Align object}
Objects in various degrees of misalignments are displayed to the user and are
asked to choose the one that is perfectly aligned.
		% Example GitHub/Kik inverted Hipop

\subsubsection{Blurred Text}
A sequence of randomly generated letters and digits are 
		presented to the user with added noise, scattered distribution and
		rotations. Sometimes, they are also presented in 3D form. 

\subsubsection{Context based}
This method is personalised to the platforms they are displayed on. They usually
pose challenges which can only be solved if the user is familiar with the
platforms. Some examples are:
	\begin{itemize}
		\item What is the name of the website's mascot?  
		\item Who owns this website?
		\item What are our members collectively called? (example: Reddit users are
			called Redditors)
	\end{itemize}

\subsubsection{Audio based}
A audio recording with added noise is presented to the user who is asked to
transcribe the content of the recording.

\subsubsection{IP tracking}
IP address is used to blacklist misbehaving users. Strictly speaking, this isn't
a CAPTCHA method but is frequently used in conjunction with other methods.

\subsubsection{Image identification}
A blurred image with added noise or unusual cropping is presented to the user
who is requested to identify the object in it. Sometimes, the users are also
asked to pick images that match a certain description from a collection of
images.

\subsubsection{Proof of Work based}
This is an alternative to CAPTCHA method that has been used for rate-limiting.
The user agent is presented with a challenge and is tasked generate a
cryptographic proof which computationally expensive.
