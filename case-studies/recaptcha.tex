\subsection{reCAPTCHA}

reCAPTCHA is a CAPTCHA system owned by Google. It is the most popular CAPTCHA
system currently deployed on the internet. The system uses the following methods
in its decision making process:

\begin{itemize}
	\item Image identification
	\item IP tracking
	\item Proprietary AI
	\item Session tracking
\end{itemize}

\subsubsection{Privacy}
Bad\\
Google's reCAPTCHA tracks its users via IP logging and session tracking. They
use supercookies to monitor their users' internet activity. The user can ban
cookies from reCAPTCHA and related services but if they did, they will be
subjected to higher difficulty puzzles or won't be allowed to access the
service. This is one of the reasons why TOR and other VPN users face
difficulties with the system.

The system also bans traffic from TOR exit nodes and due to their proprietary
and opaque decision mechanism, very little is known about how they blacklist
users.
\subsubsection{Effectiveness}
Good\\

reCAPTCHA denies access to most bots. The OCR technology used by the system is
very sophisticated. But cheap labor powered CAPTCHA farms are available which
offer CAPTCHA solving solutions for a fraction of what reCAPTCHA charges its
users. This bypass is practical as it is cheap and readily available. 

% TODO cite CAPTCHA farm cost analysis paper
ease.
\subsubsection{Accessibility}
Bad\\

reCAPTCHA was initially offering audio CAPTCHAs along with image identification
challenges but when audio recognition technology matured and was able to solve
most audio challenges, reCAPTCHA stopped offering audio challenges.

Image identification poses challenges to users with visual and cognitive
disabilities.

The IP tracking based mechanism posses accessibility threats to users behind
NATs and VPNs.

\subsubsection{Accuracy}
Bad\\

IP based tracking produces poor results when users behind NATs and VPN encounter
the service.


Overall, reCAPTCHA is a serious threat to the internet as it can, in theory,
deny access to anyone it chooses to. The decision making process is opaque and
centralised in nature and users, service providers and visitors alike, have very
little say in how the system behaves.

Also, the popularity of reCAPTCHA allows Google to track users across websites
and profile them which threatens the freedom of users on the internet.
