\subsection{mCaptcha}

mCaptcha is a free (as in freedom), experimental Proof of Work CAPTCHA system
that uses variable difficulty proofs. The AGPL licensing guarantees freedom of
its users and allows for competing implementations.

The system uses variable difficulty proofs to 
\begin{itemize}
	\item Image identification
	\item IP tracking
	\item Proprietary AI
	\item Session tracking
\end{itemize}

\subsubsection{Privacy}
Excellent\\
mCaptcha doesn't use session or IP  tracking, so it works when used in anonymous
networks like TOR\@.

\subsubsection{Effectiveness}
Excellent\\

mCaptcha uses salts to prevent rainbow table based attacks and TTLs to ban
stale entries. It also users a caching mechanism to prevent replay attacks. To
combat an attack in-progress attack, the system is designed to temporarily 
increase the difficulty of the proofs. The system also has options to
personalise proof requirements for each CAPTCHA deployment. This allows for
system administrators to select requirements which better suit their needs. 

% TODO cite CAPTCHA farm cost analysis paper
ease.
\subsubsection{Accessibility}
Good to Excellent\\

mCaptcha is non-interactive, so it is fully accessible to users with all
disabilities. To allow users on slow devices access to the service, mCaptcha has
options to set different difficulty levels. Webmasters can use this option to
set lower difficulties to while the website is receiving normal levels of
traffic and set higher levels of difficulties when the website receives unusual
amounts of traffic. Therefore, the website administrators are responsible for
mCaptcha's behaviour on their websites.

\subsubsection{Accuracy}
Excellent\\

Success and failure are absolute states in this method. A proof that fails
verification is rejected, and a successful verification is permitted. The system
is able to sense attacks and adapts by raising the difficulty requirement for
all incoming users. An attacker who is presented with increased difficulty will
either have to allocate more resources to sustain the attack or retreat in
defeat. However, this increased difficulty might impact well behaving users too,
but this is short-lived as the difficulty requirement will be relaxed once the
attack is contained.

Overall, mCaptcha is an attempt at strengthening the web and the internet with
open technologies. CAPTCHAs in their original sense were used as rate-limiters
and mCaptcha, with its PoW mechanism does just that.
